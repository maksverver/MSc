\documentclass[11pt]{article}
\usepackage{a4wide}
\title{\begin{small}MSc Project Proposal (draft)\end{small}\\
Title Goes Here}
\author{Maks Verver (0002712)\\\texttt{m.verver@student.utwente.nl}}
\date{February 3rd, 2009}
\oddsidemargin 0.0in
\topmargin 0.0in
\textwidth 6.26in
\headheight 0.0in
\begin{document}
\maketitle
Foo bar
\section{Description}
Parity games are infinite games played by two players on a finite graph. They
have a practical application in formal verification systems because parity games
corresponding to formulae expressed in the modal $\mu$-calculus can be
constructed efficiently \cite{wilke1993}.

[motivation:] Further research on parity games is important for two reasons:
\begin{enumerate}
\item Currently, formal verification tools are severely limited in the size and
complexity of models and formulae that can be verified. Better algorithms for
solving parity games could make more extensive model checking feasible.

\item The computational complexity of solving parity games is undetermined; it
is known to be in NP $\cap$ co-NP \cite{emerson2001mcmu}, but there is large gap
between this theoretical bound and practical performance (which tends to be
much better); the question whether there is a better (polynomial) time bound is
an interesting open question.
\end{enumerate}

The objectives of the project are twofold:
\begin{enumerate}
\item To assess and (if possible) improve the practical performance of parity
game algorithms in the context of formal verification, in order to increase
their usefulness as a model checking tool.

\item To gain insight in the computation process for parity game algorithms in
order to better understand the nature and complexity of solving parity games
and the associated model verification problems. Detailed analysis of the
available algorithms could lead to better understanding of the nature and
complexity of solving parity games and the associated model verification
problems.
\end{enumerate}

[approach] two promising direct algorithms using \emph{small progress measures}
\cite{jurdzinski2000spm} and using \emph{strategy improvement} \cite{voge2000dsi}.
implement in a common framework and 

"The aim of this project is to enhance understanding of the algorithms by creating an environment which allows to compare different algorithms (known from the literature) with each other, find common traits, and determine which algorithms perform better on sub-classes of the problem space."



\section{Graduation committee}
To be determined.

\section{Time schedule}
To be determined.

\bibliographystyle{alpha}
\bibliography{references}

\end{document}

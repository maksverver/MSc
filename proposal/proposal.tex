\documentclass[11pt]{article}
\usepackage{a4wide}
\title{\begin{small}MSc Project Proposal (draft)\end{small}\\
A Framework for Experimentation with Parity Games}
\author{Maks Verver (0002712)\\\texttt{m.verver@student.utwente.nl}}
\date{February 3rd, 2009}
\oddsidemargin 0.0in
\topmargin 0.0in
\textwidth 6.26in
\headheight 0.0in
\begin{document}
\maketitle

\section{Description}
Parity games are infinite games played by two players on a finite graph. They
have a practical application in formal verification systems, for example to
verify formulae expressed in the modal $\mu$-calculus \cite{wilke1993}.

Research on parity games is important for two reasons:
\begin{enumerate}
\item Currently, formal verification tools are severely limited in the size and
complexity of models and formulae that can be verified. Better algorithms for
solving parity games could make more extensive model checking feasible.

\item The computational complexity of solving parity games is undetermined. It
is known to be in NP $\cap$ co-NP \cite{emerson2001mcmu}, but there is large gap
between this theoretical bound and practical performance (which tends to be
much better). The question whether there is a better (polynomial) time bound is
an interesting open question.
\end{enumerate}

The objectives of the project are twofold:
\begin{enumerate}
\item To assess and improve the practical performance of parity game algorithms
in the context of formal verification, in order to increase their usefulness as
a model checking tool.
\item To gain insight in the computation process for parity game algorithms in
order to better understand the nature and computational complexity of parity
game algorithms and the associated model verification problems.
\end{enumerate}

To reach these objectives, it will be necessary to develop a unifying framework
for parity game solvers that allows different state-of-the-art algorithms to be
compared directly. The best known algorithms are based on
\emph{small progress measures} \cite{jurdzinski2000spm} and
\emph{strategy improvement} \cite{voge2000dsi} \cite{schewe2008osi}, which
should be included in the implementation.
This framework can then be used to assess the performance of the different algorithms
in the context of different model verification scenarios, and evaluate the
execution of these algorithms in detail. It is expected that higher performance
can be achieved with these algorithms by parallelizing their implementation,
which will be a specific approach for the project.

The concrete goals for this project are therefore:
\begin{enumerate}
\item To design and implement pluggable heuristics for the small progress measures
      algorithm, analyze existing heuristics, and devise new heuristics that
      work well in the context of model verification.
\item To redesign the existing small progress measures implementation so it is
      portable as well as parallelizable.
\item To devise an implementation of a strategy improvement algorithm that is
      competitive with the small progress measures algorithm.
\item To design and implement a portable, parallel version of the strategy
      improvement algorithm.
\item To trace the execution of the algorithms in the framework and to analyze
      the execution characteristics of the algorithms when applied to various
      inputs.
\end{enumerate}

\section{Graduation committee}
\begin{itemize}
\item First supervisor: Dr. Michael Weber (FMT)
\item Second supervisor: Prof. Dr. Jaco van de Pol (FMT)
\item External supervisor: \emph{TBD}
\end{itemize}

\section{Time period}
February 2009 -- July 2009

\bibliographystyle{alpha}
\bibliography{references}

\end{document}

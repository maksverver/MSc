\documentclass[11pt]{book}
\usepackage{a4wide}
\title{Title goes here}
\author{Maks Verver\\\texttt{m.verver@student.utwente.nl}}
\begin{document}
\maketitle


\documentclass[11pt]{book}
\usepackage{a4wide}
\title{Title goes here}
\author{Maks Verver\\\texttt{m.verver@student.utwente.nl}}
\begin{document}
\maketitle


\section{Abstract}
Blablabla.

\section{Introduction}
Intro goes here. Summarize parity games as a model checking tool, summarize
my contributions: improving practical performance of existing algortihms through
heuristics & parallelization.

The following sections give background info; none of this is new.

\section{Parity Games}
Explanation of what a parity game is; probably copying/duplication a lot of
existing definitions. Establish terminology.

\subsection{Theorethical classification}
Summarize what is known about the complexity class of solving parity games.
This gives some motivation why theoretical research on the subject is relevant.

\subsection{Application to model checking}
Explain how parity games relate to binary equation systems, mu-calculus.
This basically motivates why research into parity games is of practical
interest. Explain that improving practical performance is of interest even if
computational complexity doesn't improve. (This is critical for motivating
research into lifting heuristics and parallel algorithms.)

\subsection{Relation to other model checking approaches?}
Why would do model checking by solving parity games? What are the alternatives
and what's wrong with them?

\section{Solving Parity Games}
This details the state-of-the-art of solving parity games.
\subsection{Small Progress Measures}
- Explanation of how it works
- Importance of strategies
- Note worst-case complexity (not dependent on strategy)
\subsubsection{Lifting Strategies}
- Previously developed strategies are discussed here (linear scanning,
  predecessor lifting, focus list for parallel implementation)
\subsection{Strategy Improvement}
- Explanation of how it works
- Note worst-case complexity
\subsection{Zielonka's algorithm?}
- Not really a parity game algorithm
- Complexity (??)
\subsection{Reduction to SAT}
- Not really a parity game algorithm
- NP complete, but relatively efficient solvers available
\subsubsection{Parallelized solvers}
- Note existing non-portable SPM code
- Note parallel SAT solvers
- Note that a portable, parralel, direct solver would be novel.

Now we have covered motivation, background info, state of the art.
Nothing new yet. The rest is about my own work is hopefully new:

\section{Research Question}
This section needs a different name.

Describe globally what I'm going to do to improve the state of the art.
Mainly two parts:
(0. efficient implementation of existing algorithms
   => not really interesting on an academic level)
1. find better heuristics for small progress measures
   (detailed evaluation of SPM would be novel?)
2. find good way to parallelize small progress measures
   (this would be somewhat novel, although a lockfree implementation exists)

TBD: what am I going to do with the other algorithms?
I can't ignore them completely; at the very least I need some data for
comparison (e.g. counting of fundamental steps for the serial algorithm,
comparison with parallel SAT solver for the parallel algorithm.)

\section{Research Approach}
The purpose of this section is to describe what I did to achieve the results
in the next section, in sufficient detail so the results are reproducible (in
theory). Leave out work I did that's irrelevant to the work. Include details
only if they really matter.

\section{Theory}
This should have a different name and maybe go above Research Approach.

Write up any definitions/theorems/whatever I come up with.

\subsection{Lifting Strategies}
Any formal characterisation of lifting strategies and their properties...
Proofs e.g. of algorithm to determine minimal lifting attempts required.

\subsection{Parallelization}
Design of a parallel algorithm. Discuss different options. Motivate choices
such as (for example) chosing message passing over shared memory if such
choices are made.

\section{Framework implementation}
Explain how the framework is implemented on a high level, with details on data
structures and memory layout, insofar these may affect results. (Again, relevant
to ensure results are reproducable.)

\subsection{Serial Algorithms}
Summarize which serial algorithms were implemented (this includes a lot of stuff
others came up with for comparison).

\subsection{Lifting Strategies}
Describe which lifting strategies are actually implemented, and how to the
extend it matters.

\section{Test Data}
Describe how the test data is constructed. Include motivation.

\section{Results}
We're probably benchmarking time and memory use. Report the results.
Probably split these out into serial results and parallel results, although they
will have to be compared in the end too.

\section{Analysis}
Discuss results.

\section{Conclusions}
Quickly summarize results and explain how they relate to the research question;
describe how my research improved the state of the art (which was the research
goal). Hopefully, we can get some useful conclusion like "I've found a lifting
strategy that works well" and "I found an efficient way to solve a parity game
concurrently" (this can be quantified pretty well since we have a serial
algorithm to compare with).

\subsection{Future work}
Describe any untested hypotheses (things I thought about but don't have any
results on).
Describe possible ways to extend/improve the research.

\section{Acknowledgements}
blablabla

\bibliographystyle{alpha}
\bibliography{references}

\end{document}
